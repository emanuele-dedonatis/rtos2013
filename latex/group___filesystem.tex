\hypertarget{group___filesystem}{\section{Filesystem}
\label{group___filesystem}\index{Filesystem@{Filesystem}}
}
Filesystem related functions

File handling will not be described here since it is done using the C or C++ standard library (fopen, fprintf, istream, remove, mkdir, ...).\par
 All file related function of the C and C++ standard library should work, except rename() and link() which are not implemented.\par
 The maximum number of files that can be opened at the same time is defined in the constant M\-A\-X\-\_\-\-O\-P\-E\-N\-\_\-\-F\-I\-L\-E\-S in \hyperlink{miosix__settings_8h}{miosix/config/miosix\-\_\-settings.\-h}, All files are opened in binary mode. Therefore there is no differnce between fopen(\char`\"{}file.\-txt\char`\"{},\char`\"{}r\char`\"{}) and fopen(\char`\"{}file.\-txt\char`\"{},\char`\"{}rb\char`\"{}). For filesystem write access, the S\-Y\-N\-C\-\_\-\-A\-F\-T\-E\-R\-\_\-\-W\-R\-I\-T\-E option in \hyperlink{miosix__settings_8h}{miosix\-\_\-settings.\-h} allows to choose a faster or safer implementation.

Directory listing is not done using the standard opendir() and readdir() functions, but using the Directory class.

Mounting and unmounting the filesystem is done through the Filesystem class. Note that the filesystem is mounted automatically at boot time, except in case of errors (like no u\-S\-D card in the socket). 